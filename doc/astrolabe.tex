% astrolabe.tex
%
% The LaTeX code in this file brings together into a single document the
% various components of the model astrolabe described by Dominic Ford's paper
% in the Journal of the British Astronomical Association (2011).
%
% Copyright (C) 2010-2018 Dominic Ford <dcf21@mrao.cam.ac.uk>
%
% $Id$
%
% This code is free software; you can redistribute it and/or modify it under
% the terms of the GNU General Public License as published by the Free Software
% Foundation; either version 2 of the License, or (at your option) any later
% version.
%
% You should have received a copy of the GNU General Public License along with
% this file; if not, write to the Free Software Foundation, Inc., 51 Franklin
% Street, Fifth Floor, Boston, MA  02110-1301, USA

% ----------------------------------------------------------------------------

\documentclass[a4paper,onecolumn,10pt]{article}
\usepackage[dvips]{graphicx}
\usepackage{fancyhdr,url}
\usepackage{parskip}
\usepackage[pdftitle={Building a Model Astrolabe}, pdfauthor={Dominic Ford}, pdfsubject={The Medieval Astrolabe}, pdfkeywords={The Medieval Astrolabe}, colorlinks=true, linkcolor=blue, citecolor=blue, filecolor=blue, urlcolor=blue]{hyperref}
\renewcommand{\familydefault}{\sfdefault}
\pagestyle{fancy}

\lhead{\it MAKE YOUR OWN MODEL ASTROLABE}
\chead{}
\rhead{\thepage}
\lfoot{}\rfoot{}
\cfoot{\bf\footnotesize\copyright\ 2010--2018 Dominic Ford. Distributed under the GNU General Public License, version 3. Document downloaded from \url{https://in-the-sky.org/astrolabe/index.html}}

\fancypagestyle{plain}{%
\fancyhf{} % clear all header and footer fields
\renewcommand{\headrulewidth}{0pt}
\renewcommand{\footrulewidth}{0pt}}

\title{Make your own Model Astrolabe}
\author{Dominic Ford}
\date{2010--2018}

\addtolength{\topmargin}{-.3in}
\addtolength{\textheight}{.6in}

\begin{document}
\maketitle
\setcounter{footnote}{1}

An astrolabe is an elaborate instrument which combines a mechanical model of
the sky's rotation through the night -- similar to a modern planisphere
-- with an observing instrument which allows the altitudes of objects in
the night sky to be measured. Put together, these two components can be used
the time of day, by determining the altitude of a star, and also at what time
of day the sky's rotation brings it to that height above the horizon.

Historically, the astrolabe was the most sophisticated astronomical instrument
in widespread use in the Middle Ages.  In fact, it held this position for
nearly two thousand years, from the time of Hipparchus (c. 190--120 BCE)
until the turn of the seventeenth century, around the time that the telescope
was invented in 1609.

Yet today this complex instrument is rarely seen outside of glass cases in
museums, and those interested in learning about it may have some difficulty
finding a specimen to play with. Ornately carved brass reproductions are
available from some telescope dealers, but with substantial price tags
attached. These price tags are historically authentic: medieval astrolabes
were often made from high-cost materials and intricately decorated, becoming
expensive items of beauty as well as practical observing instruments. But for
the amateur astronomer who is looking for a toy with which to muse over past
observing practice, a simpler alternative may be preferable.

This document provides a cardboard cut-and-glue kit to make your own model
astrolabe, so that you can rediscover medieval observing practice for yourself.
The astrolabe presented here is design for use at a latitude of
\input{tmp/lat}. Alternative versions, prepared for other latitudes, are
available from the author's website:

\centerline{\url{https://in-the-sky.org/astrolabe/}}

\section*{Assembly Instructions}

To build a model astrolabe tailored for a latitude of \input{tmp/lat},
Figures~\ref{mother_back}, \ref{mother_front} and~\ref{rule} should be printed
out onto paper, or more preferably onto thin card.  Figure~\ref{rete} should be
printed onto a sheet of transparent acetate.  The two sides of the {\it mother}
(Figures~\ref{mother_back} and~\ref{mother_front}) should be glued rigidly
back-to-back, perhaps sandwiching a piece of rigid card. The {\it rete},
printed onto transparent acetate\footnote{Historically, the rete would have
been made of the same material as the rest of the astrolabe and marked with
arrows showing the positions of prominent stars. As much of the material of the
rete as possible would then have been cut away to allow the climate below to be
seen. We use transparent plastic here because it is so much more practical than
the traditional form of rete.}, should be placed over the {\it climate}, which
for simplicity is incorporated into the front of the mother in this document.

The {\it rule} and the {\it alidade} should be placed on either side of the
astrolabe: the rule, marked out with a declination scale, should rotate over
the front of the mother; the alidade should rotate over the back of the mother.
The two tabs on the side of the alidade should be folded out to form a sight
used for measuring the altitudes of celestial and terrestrial objects.  The
whole construction may then finally be fastened together by placing a split-pin
paper fastener through the centre.

\section*{How to use an astrolabe}

For more information about how to use your astrolabe, see the author's website,

\centerline{\url{https://in-the-sky.org/astrolabe/}}

or download a copy of the author's paper about the astrolabe, which was published in the Journal of the British Astronomical Association, and can be downloaded here:

\centerline{\url{https://in-the-sky.org/astrolabe/astrolabe\_jbaa.pdf}}

\section*{Customised Astrolabes}

The astrolabe images presented here were produced using PyXPlot, an open-source
vector graphics scripting language developed by the same author.  PyXPlot has a
website\footnote{\url{http://www.pyxplot.org.uk}} with extensive documentation,
and is available as a standard package in a number of Linux distributions
including Ubuntu, Debian and Gentoo. Unfortunately, it is not available for
Microsoft Windows at the present time.

The PyXPlot scripts used to generate the images in this document are included
in the accompanying file archive and may be modified to generate customised
astrolabes. For example, to produce an astrolabe with your own choice of
saints' days or birthdays on the back of the mother, the file {\tt
Raw\-Data/\-Saints\-Days\-.dat} should be modified. A {\tt python} script, {\tt
main.py}, is included which rebuilds all of the image files shipped in the {\tt
astrolabe\_parts} folder.

\begin{thebibliography}{9}
\bibitem{Ford}Ford, D.C., \textit{J.\ Brit.\ astr.\ Ass.}, 131(1), 33 (2012).
\bibitem{chaucer}Chaucer, G., \textit{Treatise on the Astrolabe}, in {\it The Riverside Chaucer}, ed.\ L.D.\ Benson (Boston, 1987)
\bibitem{pap1}Eisner, S., \textit{J.\ Brit.\ astr.\ Ass.}, \textbf{86}(1), 18-29 (1975)
\bibitem{pap2}Eisner, S., \textit{J.\ Brit.\ astr.\ Ass.}, \textbf{86}(2), 125-132 (1976a)
\bibitem{pap3}Eisner, S., \textit{J.\ Brit.\ astr.\ Ass.}, \textbf{86}(3), 219-227 (1976b)
\end{thebibliography}

\newpage

\begin{figure}
\centerline{\includegraphics{tmp/mother_back}}
\caption{The back of the mother of the astrolabe.}
\label{mother_back}
\end{figure}

\begin{figure}
\centerline{\includegraphics{tmp/mother_front}}
\caption{The front of the mother of the astrolabe, with combined climate.
Should a climate for a different latitude be required, the accompanying file archive should be downloaded. This includes alternative versions of this document for any latitude on the Earth at $5^\circ$ intervals.}
\label{mother_front}
\end{figure}

\begin{figure}
\centerline{\includegraphics{tmp/rule}}
\caption{Left: The rule, which should be mounted on the front of the astrolabe. Right: The alidade, which should be mounted on the back of the astrolabe.}
\label{rule}
\end{figure}

\begin{figure}
\centerline{\includegraphics{tmp/rete}}
\caption{The rete of the astrolabe, showing the stars of the night sky. This should be printed onto a piece of transparent plastic; most stationers should be able to provide acetate sheets for use on overhead projectors, which are ideal for this purpose.}
\label{rete}
\end{figure}

\end{document}

